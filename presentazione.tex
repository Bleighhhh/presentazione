\documentclass{beamer}

\title{Business model innovation for sustainability \\ A Value Change For Survival}
\author{Pietro Mecca}

#10 minuti di presentazione: 12-15 slides 

\begin{document}

\maketitle

\begin{frame}
  \frametitle{Research purpose} % perchè ho fatto la ricerca 

  \begin{itemize}
    \item Background
    \item Goal
    \item Solution
    \item Analysis
  \end{itemize}
\end{frame}

\begin{frame}
  \frametitle{Research questions}  %research questions

  \begin{itemize}
    \item New sports car is being launched
    \begin{itemize}
     \item 2 Models: priced from €60k to €80k
    \end{itemize}
    \item Buyers allocate 20\% of their income to these vehicles purchase
    \item AutoDep wants to open an office in Europe
    \item Candidate cities are: Paris, Zurich, Geneva, Berlin
  \end{itemize}
\end{frame}

\begin{frame}
  \frametitle{Goal} % metodologia, da stressare insieme alle interviste che ho fatto e perchè, magari ricordando che alcune le ho potute fare solamente perchè ero in Svezia in doppia laurea

  \begin{itemize}
    \item The goal is to identify the most potentially profitable city
  \end{itemize}
\end{frame}

\begin{frame}
  \frametitle{Results} % risultati, magari se si riesce a riportare qualche frase ad effetto è ottimo 
1- non esistono i business model, sono cambiati ogni giorno a lavoro
2- business model innovation è tweaking, ma ha diversi risultati, business  model può essere diverso ma non nuovo e nuovo ma non sostenibile, ad esempio Amazon (la distinzione potrebbe essere simile a quella proposta insimee alla definzione di disruptive innovation)
3. La sostenibilità non può essere fatta a metà. ma anzi serve un equilibrio tra tutti i suoi componenti (tuti ma sopratgtutto hakansson)
4. la sosttenibilità varia a seconda delle tecnologie (ytterbon) ed è legata ai valori, etici, che si evolvono
5. porta a migliori prestazioni per svariati motiviu (Evelina, Ytterbon, Hassle, Olivieri, Hakannson)
  \begin{itemize}
    \item Zurich is the most profitable city with an expected profit of €190k.\\
    Berlin follows, third is Paris and last is Geneva
  \end{itemize}
\end{frame}

\begin{frame}
  \frametitle{Conclusions} %conclusioni

  1. conclusioni tipo "sosteinbilità si definisce così; il rapporto logico tra i cosi è questo e la risposta alla domanda è questa"

  2. proposizione del nuovo framework teorico 
 \textbf{total annual expected profit per city} = \\
 total annual expected revenue - annual costs \\
\end{frame}


\end{document}
